
% Default to the notebook output style

    


% Inherit from the specified cell style.




    
\documentclass[11pt]{article}

    
    
    \usepackage[T1]{fontenc}
    % Nicer default font (+ math font) than Computer Modern for most use cases
    \usepackage{mathpazo}

    % Basic figure setup, for now with no caption control since it's done
    % automatically by Pandoc (which extracts ![](path) syntax from Markdown).
    \usepackage{graphicx}
    % We will generate all images so they have a width \maxwidth. This means
    % that they will get their normal width if they fit onto the page, but
    % are scaled down if they would overflow the margins.
    \makeatletter
    \def\maxwidth{\ifdim\Gin@nat@width>\linewidth\linewidth
    \else\Gin@nat@width\fi}
    \makeatother
    \let\Oldincludegraphics\includegraphics
    % Set max figure width to be 80% of text width, for now hardcoded.
    \renewcommand{\includegraphics}[1]{\Oldincludegraphics[width=.8\maxwidth]{#1}}
    % Ensure that by default, figures have no caption (until we provide a
    % proper Figure object with a Caption API and a way to capture that
    % in the conversion process - todo).
    \usepackage{caption}
    \DeclareCaptionLabelFormat{nolabel}{}
    \captionsetup{labelformat=nolabel}

    \usepackage{adjustbox} % Used to constrain images to a maximum size 
    \usepackage{xcolor} % Allow colors to be defined
    \usepackage{enumerate} % Needed for markdown enumerations to work
    \usepackage{geometry} % Used to adjust the document margins
    \usepackage{amsmath} % Equations
    \usepackage{amssymb} % Equations
    \usepackage{textcomp} % defines textquotesingle
    % Hack from http://tex.stackexchange.com/a/47451/13684:
    \AtBeginDocument{%
        \def\PYZsq{\textquotesingle}% Upright quotes in Pygmentized code
    }
    \usepackage{upquote} % Upright quotes for verbatim code
    \usepackage{eurosym} % defines \euro
    \usepackage[mathletters]{ucs} % Extended unicode (utf-8) support
    \usepackage[utf8x]{inputenc} % Allow utf-8 characters in the tex document
    \usepackage{fancyvrb} % verbatim replacement that allows latex
    \usepackage{grffile} % extends the file name processing of package graphics 
                         % to support a larger range 
    % The hyperref package gives us a pdf with properly built
    % internal navigation ('pdf bookmarks' for the table of contents,
    % internal cross-reference links, web links for URLs, etc.)
    \usepackage{hyperref}
    \usepackage{longtable} % longtable support required by pandoc >1.10
    \usepackage{booktabs}  % table support for pandoc > 1.12.2
    \usepackage[inline]{enumitem} % IRkernel/repr support (it uses the enumerate* environment)
    \usepackage[normalem]{ulem} % ulem is needed to support strikethroughs (\sout)
                                % normalem makes italics be italics, not underlines
    

    
    
    % Colors for the hyperref package
    \definecolor{urlcolor}{rgb}{0,.145,.698}
    \definecolor{linkcolor}{rgb}{.71,0.21,0.01}
    \definecolor{citecolor}{rgb}{.12,.54,.11}

    % ANSI colors
    \definecolor{ansi-black}{HTML}{3E424D}
    \definecolor{ansi-black-intense}{HTML}{282C36}
    \definecolor{ansi-red}{HTML}{E75C58}
    \definecolor{ansi-red-intense}{HTML}{B22B31}
    \definecolor{ansi-green}{HTML}{00A250}
    \definecolor{ansi-green-intense}{HTML}{007427}
    \definecolor{ansi-yellow}{HTML}{DDB62B}
    \definecolor{ansi-yellow-intense}{HTML}{B27D12}
    \definecolor{ansi-blue}{HTML}{208FFB}
    \definecolor{ansi-blue-intense}{HTML}{0065CA}
    \definecolor{ansi-magenta}{HTML}{D160C4}
    \definecolor{ansi-magenta-intense}{HTML}{A03196}
    \definecolor{ansi-cyan}{HTML}{60C6C8}
    \definecolor{ansi-cyan-intense}{HTML}{258F8F}
    \definecolor{ansi-white}{HTML}{C5C1B4}
    \definecolor{ansi-white-intense}{HTML}{A1A6B2}

    % commands and environments needed by pandoc snippets
    % extracted from the output of `pandoc -s`
    \providecommand{\tightlist}{%
      \setlength{\itemsep}{0pt}\setlength{\parskip}{0pt}}
    \DefineVerbatimEnvironment{Highlighting}{Verbatim}{commandchars=\\\{\}}
    % Add ',fontsize=\small' for more characters per line
    \newenvironment{Shaded}{}{}
    \newcommand{\KeywordTok}[1]{\textcolor[rgb]{0.00,0.44,0.13}{\textbf{{#1}}}}
    \newcommand{\DataTypeTok}[1]{\textcolor[rgb]{0.56,0.13,0.00}{{#1}}}
    \newcommand{\DecValTok}[1]{\textcolor[rgb]{0.25,0.63,0.44}{{#1}}}
    \newcommand{\BaseNTok}[1]{\textcolor[rgb]{0.25,0.63,0.44}{{#1}}}
    \newcommand{\FloatTok}[1]{\textcolor[rgb]{0.25,0.63,0.44}{{#1}}}
    \newcommand{\CharTok}[1]{\textcolor[rgb]{0.25,0.44,0.63}{{#1}}}
    \newcommand{\StringTok}[1]{\textcolor[rgb]{0.25,0.44,0.63}{{#1}}}
    \newcommand{\CommentTok}[1]{\textcolor[rgb]{0.38,0.63,0.69}{\textit{{#1}}}}
    \newcommand{\OtherTok}[1]{\textcolor[rgb]{0.00,0.44,0.13}{{#1}}}
    \newcommand{\AlertTok}[1]{\textcolor[rgb]{1.00,0.00,0.00}{\textbf{{#1}}}}
    \newcommand{\FunctionTok}[1]{\textcolor[rgb]{0.02,0.16,0.49}{{#1}}}
    \newcommand{\RegionMarkerTok}[1]{{#1}}
    \newcommand{\ErrorTok}[1]{\textcolor[rgb]{1.00,0.00,0.00}{\textbf{{#1}}}}
    \newcommand{\NormalTok}[1]{{#1}}
    
    % Additional commands for more recent versions of Pandoc
    \newcommand{\ConstantTok}[1]{\textcolor[rgb]{0.53,0.00,0.00}{{#1}}}
    \newcommand{\SpecialCharTok}[1]{\textcolor[rgb]{0.25,0.44,0.63}{{#1}}}
    \newcommand{\VerbatimStringTok}[1]{\textcolor[rgb]{0.25,0.44,0.63}{{#1}}}
    \newcommand{\SpecialStringTok}[1]{\textcolor[rgb]{0.73,0.40,0.53}{{#1}}}
    \newcommand{\ImportTok}[1]{{#1}}
    \newcommand{\DocumentationTok}[1]{\textcolor[rgb]{0.73,0.13,0.13}{\textit{{#1}}}}
    \newcommand{\AnnotationTok}[1]{\textcolor[rgb]{0.38,0.63,0.69}{\textbf{\textit{{#1}}}}}
    \newcommand{\CommentVarTok}[1]{\textcolor[rgb]{0.38,0.63,0.69}{\textbf{\textit{{#1}}}}}
    \newcommand{\VariableTok}[1]{\textcolor[rgb]{0.10,0.09,0.49}{{#1}}}
    \newcommand{\ControlFlowTok}[1]{\textcolor[rgb]{0.00,0.44,0.13}{\textbf{{#1}}}}
    \newcommand{\OperatorTok}[1]{\textcolor[rgb]{0.40,0.40,0.40}{{#1}}}
    \newcommand{\BuiltInTok}[1]{{#1}}
    \newcommand{\ExtensionTok}[1]{{#1}}
    \newcommand{\PreprocessorTok}[1]{\textcolor[rgb]{0.74,0.48,0.00}{{#1}}}
    \newcommand{\AttributeTok}[1]{\textcolor[rgb]{0.49,0.56,0.16}{{#1}}}
    \newcommand{\InformationTok}[1]{\textcolor[rgb]{0.38,0.63,0.69}{\textbf{\textit{{#1}}}}}
    \newcommand{\WarningTok}[1]{\textcolor[rgb]{0.38,0.63,0.69}{\textbf{\textit{{#1}}}}}
    
    
    % Define a nice break command that doesn't care if a line doesn't already
    % exist.
    \def\br{\hspace*{\fill} \\* }
    % Math Jax compatability definitions
    \def\gt{>}
    \def\lt{<}
    % Document parameters
    \title{Lecture 6 Notebook}
    
    
    

    % Pygments definitions
    
\makeatletter
\def\PY@reset{\let\PY@it=\relax \let\PY@bf=\relax%
    \let\PY@ul=\relax \let\PY@tc=\relax%
    \let\PY@bc=\relax \let\PY@ff=\relax}
\def\PY@tok#1{\csname PY@tok@#1\endcsname}
\def\PY@toks#1+{\ifx\relax#1\empty\else%
    \PY@tok{#1}\expandafter\PY@toks\fi}
\def\PY@do#1{\PY@bc{\PY@tc{\PY@ul{%
    \PY@it{\PY@bf{\PY@ff{#1}}}}}}}
\def\PY#1#2{\PY@reset\PY@toks#1+\relax+\PY@do{#2}}

\expandafter\def\csname PY@tok@w\endcsname{\def\PY@tc##1{\textcolor[rgb]{0.73,0.73,0.73}{##1}}}
\expandafter\def\csname PY@tok@c\endcsname{\let\PY@it=\textit\def\PY@tc##1{\textcolor[rgb]{0.25,0.50,0.50}{##1}}}
\expandafter\def\csname PY@tok@cp\endcsname{\def\PY@tc##1{\textcolor[rgb]{0.74,0.48,0.00}{##1}}}
\expandafter\def\csname PY@tok@k\endcsname{\let\PY@bf=\textbf\def\PY@tc##1{\textcolor[rgb]{0.00,0.50,0.00}{##1}}}
\expandafter\def\csname PY@tok@kp\endcsname{\def\PY@tc##1{\textcolor[rgb]{0.00,0.50,0.00}{##1}}}
\expandafter\def\csname PY@tok@kt\endcsname{\def\PY@tc##1{\textcolor[rgb]{0.69,0.00,0.25}{##1}}}
\expandafter\def\csname PY@tok@o\endcsname{\def\PY@tc##1{\textcolor[rgb]{0.40,0.40,0.40}{##1}}}
\expandafter\def\csname PY@tok@ow\endcsname{\let\PY@bf=\textbf\def\PY@tc##1{\textcolor[rgb]{0.67,0.13,1.00}{##1}}}
\expandafter\def\csname PY@tok@nb\endcsname{\def\PY@tc##1{\textcolor[rgb]{0.00,0.50,0.00}{##1}}}
\expandafter\def\csname PY@tok@nf\endcsname{\def\PY@tc##1{\textcolor[rgb]{0.00,0.00,1.00}{##1}}}
\expandafter\def\csname PY@tok@nc\endcsname{\let\PY@bf=\textbf\def\PY@tc##1{\textcolor[rgb]{0.00,0.00,1.00}{##1}}}
\expandafter\def\csname PY@tok@nn\endcsname{\let\PY@bf=\textbf\def\PY@tc##1{\textcolor[rgb]{0.00,0.00,1.00}{##1}}}
\expandafter\def\csname PY@tok@ne\endcsname{\let\PY@bf=\textbf\def\PY@tc##1{\textcolor[rgb]{0.82,0.25,0.23}{##1}}}
\expandafter\def\csname PY@tok@nv\endcsname{\def\PY@tc##1{\textcolor[rgb]{0.10,0.09,0.49}{##1}}}
\expandafter\def\csname PY@tok@no\endcsname{\def\PY@tc##1{\textcolor[rgb]{0.53,0.00,0.00}{##1}}}
\expandafter\def\csname PY@tok@nl\endcsname{\def\PY@tc##1{\textcolor[rgb]{0.63,0.63,0.00}{##1}}}
\expandafter\def\csname PY@tok@ni\endcsname{\let\PY@bf=\textbf\def\PY@tc##1{\textcolor[rgb]{0.60,0.60,0.60}{##1}}}
\expandafter\def\csname PY@tok@na\endcsname{\def\PY@tc##1{\textcolor[rgb]{0.49,0.56,0.16}{##1}}}
\expandafter\def\csname PY@tok@nt\endcsname{\let\PY@bf=\textbf\def\PY@tc##1{\textcolor[rgb]{0.00,0.50,0.00}{##1}}}
\expandafter\def\csname PY@tok@nd\endcsname{\def\PY@tc##1{\textcolor[rgb]{0.67,0.13,1.00}{##1}}}
\expandafter\def\csname PY@tok@s\endcsname{\def\PY@tc##1{\textcolor[rgb]{0.73,0.13,0.13}{##1}}}
\expandafter\def\csname PY@tok@sd\endcsname{\let\PY@it=\textit\def\PY@tc##1{\textcolor[rgb]{0.73,0.13,0.13}{##1}}}
\expandafter\def\csname PY@tok@si\endcsname{\let\PY@bf=\textbf\def\PY@tc##1{\textcolor[rgb]{0.73,0.40,0.53}{##1}}}
\expandafter\def\csname PY@tok@se\endcsname{\let\PY@bf=\textbf\def\PY@tc##1{\textcolor[rgb]{0.73,0.40,0.13}{##1}}}
\expandafter\def\csname PY@tok@sr\endcsname{\def\PY@tc##1{\textcolor[rgb]{0.73,0.40,0.53}{##1}}}
\expandafter\def\csname PY@tok@ss\endcsname{\def\PY@tc##1{\textcolor[rgb]{0.10,0.09,0.49}{##1}}}
\expandafter\def\csname PY@tok@sx\endcsname{\def\PY@tc##1{\textcolor[rgb]{0.00,0.50,0.00}{##1}}}
\expandafter\def\csname PY@tok@m\endcsname{\def\PY@tc##1{\textcolor[rgb]{0.40,0.40,0.40}{##1}}}
\expandafter\def\csname PY@tok@gh\endcsname{\let\PY@bf=\textbf\def\PY@tc##1{\textcolor[rgb]{0.00,0.00,0.50}{##1}}}
\expandafter\def\csname PY@tok@gu\endcsname{\let\PY@bf=\textbf\def\PY@tc##1{\textcolor[rgb]{0.50,0.00,0.50}{##1}}}
\expandafter\def\csname PY@tok@gd\endcsname{\def\PY@tc##1{\textcolor[rgb]{0.63,0.00,0.00}{##1}}}
\expandafter\def\csname PY@tok@gi\endcsname{\def\PY@tc##1{\textcolor[rgb]{0.00,0.63,0.00}{##1}}}
\expandafter\def\csname PY@tok@gr\endcsname{\def\PY@tc##1{\textcolor[rgb]{1.00,0.00,0.00}{##1}}}
\expandafter\def\csname PY@tok@ge\endcsname{\let\PY@it=\textit}
\expandafter\def\csname PY@tok@gs\endcsname{\let\PY@bf=\textbf}
\expandafter\def\csname PY@tok@gp\endcsname{\let\PY@bf=\textbf\def\PY@tc##1{\textcolor[rgb]{0.00,0.00,0.50}{##1}}}
\expandafter\def\csname PY@tok@go\endcsname{\def\PY@tc##1{\textcolor[rgb]{0.53,0.53,0.53}{##1}}}
\expandafter\def\csname PY@tok@gt\endcsname{\def\PY@tc##1{\textcolor[rgb]{0.00,0.27,0.87}{##1}}}
\expandafter\def\csname PY@tok@err\endcsname{\def\PY@bc##1{\setlength{\fboxsep}{0pt}\fcolorbox[rgb]{1.00,0.00,0.00}{1,1,1}{\strut ##1}}}
\expandafter\def\csname PY@tok@kc\endcsname{\let\PY@bf=\textbf\def\PY@tc##1{\textcolor[rgb]{0.00,0.50,0.00}{##1}}}
\expandafter\def\csname PY@tok@kd\endcsname{\let\PY@bf=\textbf\def\PY@tc##1{\textcolor[rgb]{0.00,0.50,0.00}{##1}}}
\expandafter\def\csname PY@tok@kn\endcsname{\let\PY@bf=\textbf\def\PY@tc##1{\textcolor[rgb]{0.00,0.50,0.00}{##1}}}
\expandafter\def\csname PY@tok@kr\endcsname{\let\PY@bf=\textbf\def\PY@tc##1{\textcolor[rgb]{0.00,0.50,0.00}{##1}}}
\expandafter\def\csname PY@tok@bp\endcsname{\def\PY@tc##1{\textcolor[rgb]{0.00,0.50,0.00}{##1}}}
\expandafter\def\csname PY@tok@fm\endcsname{\def\PY@tc##1{\textcolor[rgb]{0.00,0.00,1.00}{##1}}}
\expandafter\def\csname PY@tok@vc\endcsname{\def\PY@tc##1{\textcolor[rgb]{0.10,0.09,0.49}{##1}}}
\expandafter\def\csname PY@tok@vg\endcsname{\def\PY@tc##1{\textcolor[rgb]{0.10,0.09,0.49}{##1}}}
\expandafter\def\csname PY@tok@vi\endcsname{\def\PY@tc##1{\textcolor[rgb]{0.10,0.09,0.49}{##1}}}
\expandafter\def\csname PY@tok@vm\endcsname{\def\PY@tc##1{\textcolor[rgb]{0.10,0.09,0.49}{##1}}}
\expandafter\def\csname PY@tok@sa\endcsname{\def\PY@tc##1{\textcolor[rgb]{0.73,0.13,0.13}{##1}}}
\expandafter\def\csname PY@tok@sb\endcsname{\def\PY@tc##1{\textcolor[rgb]{0.73,0.13,0.13}{##1}}}
\expandafter\def\csname PY@tok@sc\endcsname{\def\PY@tc##1{\textcolor[rgb]{0.73,0.13,0.13}{##1}}}
\expandafter\def\csname PY@tok@dl\endcsname{\def\PY@tc##1{\textcolor[rgb]{0.73,0.13,0.13}{##1}}}
\expandafter\def\csname PY@tok@s2\endcsname{\def\PY@tc##1{\textcolor[rgb]{0.73,0.13,0.13}{##1}}}
\expandafter\def\csname PY@tok@sh\endcsname{\def\PY@tc##1{\textcolor[rgb]{0.73,0.13,0.13}{##1}}}
\expandafter\def\csname PY@tok@s1\endcsname{\def\PY@tc##1{\textcolor[rgb]{0.73,0.13,0.13}{##1}}}
\expandafter\def\csname PY@tok@mb\endcsname{\def\PY@tc##1{\textcolor[rgb]{0.40,0.40,0.40}{##1}}}
\expandafter\def\csname PY@tok@mf\endcsname{\def\PY@tc##1{\textcolor[rgb]{0.40,0.40,0.40}{##1}}}
\expandafter\def\csname PY@tok@mh\endcsname{\def\PY@tc##1{\textcolor[rgb]{0.40,0.40,0.40}{##1}}}
\expandafter\def\csname PY@tok@mi\endcsname{\def\PY@tc##1{\textcolor[rgb]{0.40,0.40,0.40}{##1}}}
\expandafter\def\csname PY@tok@il\endcsname{\def\PY@tc##1{\textcolor[rgb]{0.40,0.40,0.40}{##1}}}
\expandafter\def\csname PY@tok@mo\endcsname{\def\PY@tc##1{\textcolor[rgb]{0.40,0.40,0.40}{##1}}}
\expandafter\def\csname PY@tok@ch\endcsname{\let\PY@it=\textit\def\PY@tc##1{\textcolor[rgb]{0.25,0.50,0.50}{##1}}}
\expandafter\def\csname PY@tok@cm\endcsname{\let\PY@it=\textit\def\PY@tc##1{\textcolor[rgb]{0.25,0.50,0.50}{##1}}}
\expandafter\def\csname PY@tok@cpf\endcsname{\let\PY@it=\textit\def\PY@tc##1{\textcolor[rgb]{0.25,0.50,0.50}{##1}}}
\expandafter\def\csname PY@tok@c1\endcsname{\let\PY@it=\textit\def\PY@tc##1{\textcolor[rgb]{0.25,0.50,0.50}{##1}}}
\expandafter\def\csname PY@tok@cs\endcsname{\let\PY@it=\textit\def\PY@tc##1{\textcolor[rgb]{0.25,0.50,0.50}{##1}}}

\def\PYZbs{\char`\\}
\def\PYZus{\char`\_}
\def\PYZob{\char`\{}
\def\PYZcb{\char`\}}
\def\PYZca{\char`\^}
\def\PYZam{\char`\&}
\def\PYZlt{\char`\<}
\def\PYZgt{\char`\>}
\def\PYZsh{\char`\#}
\def\PYZpc{\char`\%}
\def\PYZdl{\char`\$}
\def\PYZhy{\char`\-}
\def\PYZsq{\char`\'}
\def\PYZdq{\char`\"}
\def\PYZti{\char`\~}
% for compatibility with earlier versions
\def\PYZat{@}
\def\PYZlb{[}
\def\PYZrb{]}
\makeatother


    % Exact colors from NB
    \definecolor{incolor}{rgb}{0.0, 0.0, 0.5}
    \definecolor{outcolor}{rgb}{0.545, 0.0, 0.0}



    
    % Prevent overflowing lines due to hard-to-break entities
    \sloppy 
    % Setup hyperref package
    \hypersetup{
      breaklinks=true,  % so long urls are correctly broken across lines
      colorlinks=true,
      urlcolor=urlcolor,
      linkcolor=linkcolor,
      citecolor=citecolor,
      }
    % Slightly bigger margins than the latex defaults
    
    \geometry{verbose,tmargin=1in,bmargin=1in,lmargin=1in,rmargin=1in}
    
    

    \begin{document}
    
    
    \maketitle
    
    

    
    \section{ER190C Lecture 6 Notebook}\label{er190c-lecture-6-notebook}

\textbf{Data Cleaning and Exploratory Data Analysis}

Duncan Callaway

September 11 2018

Today we'll work with PurpleAir data to explore the concepts of
Structure, Granularity, Scope, Temporality and Faithfulness. Along the
way we'll talk about data cleaning as well.

\href{https://www.purpleair.com/map\#1/25/-30}{Here's PurpleAir's
website} -\/- They have really cool maps!

The way I developed this lecture was by pulling the data down and
exploring it. You'll see my (edited) process of examining the data.

This began by me visiting
\href{https://www.purpleair.com/sensorlist}{this website} to look for
data. I used the Chrome browser to pull data (other browsers didn't
work).

The folks are PurpleAir also sent me
\href{https://github.com/ds-modules/ER-190C/blob/master/lecture/Lecture\%206\%20Sept\%2011/Using\%20PurpleAir\%20Data.pdf}{this
pdf} describing their data.

    \begin{Verbatim}[commandchars=\\\{\}]
{\color{incolor}In [{\color{incolor}1}]:} \PY{k+kn}{import} \PY{n+nn}{numpy} \PY{k}{as} \PY{n+nn}{np}
        \PY{k+kn}{import} \PY{n+nn}{pandas} \PY{k}{as} \PY{n+nn}{pd}
        \PY{k+kn}{import} \PY{n+nn}{os}
\end{Verbatim}


    \subsection{Structure.}\label{structure.}

First let's look at what's in the data directory using
\texttt{os.listdir} (remember this is a set of command line-style
commands that work across platforms, i.e. mac, linux, windows)

    \begin{Verbatim}[commandchars=\\\{\}]
{\color{incolor}In [{\color{incolor}2}]:} \PY{n}{os}\PY{o}{.}\PY{n}{listdir}\PY{p}{(}\PY{l+s+s1}{\PYZsq{}}\PY{l+s+s1}{data}\PY{l+s+s1}{\PYZsq{}}\PY{p}{)}
\end{Verbatim}


\begin{Verbatim}[commandchars=\\\{\}]
{\color{outcolor}Out[{\color{outcolor}2}]:} ['.DS\_Store',
         'Ecole Bilingue de Berkeley (37.854830799999995 -122.28937169999999) Primary 08\_05\_2018 09\_04\_2018.csv',
         'Ecole Bilingue de Berkeley (37.854830799999995 -122.28937169999999) Secondary 08\_05\_2018 09\_04\_2018.csv']
\end{Verbatim}
            
    What can we learn from these file names? * the sensor location is
probably the French School in Berkeley. * Looks like lat / lon
coordinates in parens * the date range is listed * there is a secondary
/ primary distinction.

Before proceeding let's find the size of some of these files:

    \begin{Verbatim}[commandchars=\\\{\}]
{\color{incolor}In [{\color{incolor}3}]:} \PY{n}{os}\PY{o}{.}\PY{n}{path}\PY{o}{.}\PY{n}{getsize}\PY{p}{(}\PY{l+s+s1}{\PYZsq{}}\PY{l+s+s1}{data/Ecole Bilingue de Berkeley (37.854830799999995 \PYZhy{}122.28937169999999) Primary 08\PYZus{}05\PYZus{}2018 09\PYZus{}04\PYZus{}2018.csv}\PY{l+s+s1}{\PYZsq{}}\PY{p}{)}
\end{Verbatim}


\begin{Verbatim}[commandchars=\\\{\}]
{\color{outcolor}Out[{\color{outcolor}3}]:} 2381187
\end{Verbatim}
            
    What are the units? Let's shift tab in to \texttt{getsize} to find out.

    \begin{Verbatim}[commandchars=\\\{\}]
{\color{incolor}In [{\color{incolor}4}]:} \PY{n}{os}\PY{o}{.}\PY{n}{path}\PY{o}{.}\PY{n}{getsize}
\end{Verbatim}


\begin{Verbatim}[commandchars=\\\{\}]
{\color{outcolor}Out[{\color{outcolor}4}]:} <function genericpath.getsize>
\end{Verbatim}
            
    Not much information. Google search reveals
\href{https://docs.python.org/2/library/os.path.html}{this} information
page, which says the units are bytes.

    \begin{Verbatim}[commandchars=\\\{\}]
{\color{incolor}In [{\color{incolor}5}]:} \PY{n}{os}\PY{o}{.}\PY{n}{path}\PY{o}{.}\PY{n}{getsize}\PY{p}{(}\PY{l+s+s1}{\PYZsq{}}\PY{l+s+s1}{data/Ecole Bilingue de Berkeley (37.854830799999995 \PYZhy{}122.28937169999999) Primary 08\PYZus{}05\PYZus{}2018 09\PYZus{}04\PYZus{}2018.csv}\PY{l+s+s1}{\PYZsq{}}\PY{p}{)}\PY{o}{/}\PY{l+m+mf}{1e6}
\end{Verbatim}


\begin{Verbatim}[commandchars=\\\{\}]
{\color{outcolor}Out[{\color{outcolor}5}]:} 2.381187
\end{Verbatim}
            
    SO 2.4 Mb.

    \begin{Verbatim}[commandchars=\\\{\}]
{\color{incolor}In [{\color{incolor}6}]:} \PY{n}{os}\PY{o}{.}\PY{n}{path}\PY{o}{.}\PY{n}{getsize}\PY{p}{(}\PY{l+s+s1}{\PYZsq{}}\PY{l+s+s1}{data/Ecole Bilingue de Berkeley (37.854830799999995 \PYZhy{}122.28937169999999) Secondary 08\PYZus{}05\PYZus{}2018 09\PYZus{}04\PYZus{}2018.csv}\PY{l+s+s1}{\PYZsq{}}\PY{p}{)}\PY{o}{/}\PY{l+m+mf}{1e6}
\end{Verbatim}


\begin{Verbatim}[commandchars=\\\{\}]
{\color{outcolor}Out[{\color{outcolor}6}]:} 2.497975
\end{Verbatim}
            
    Before we go furhter, what's the primary vs secondary data file?

Checking out the "Using Purple Air data" pdf, provided to my by them, it
looks like the two files contain different data. We'll focus on PM2.5,
which is in the primary file.

    In this directory there is a python file (\texttt{utils.py}) that has
some useful utilities -\/- we'll pull some in over the course of the
lecture. First to use is \texttt{line\_count}

    \begin{Verbatim}[commandchars=\\\{\}]
{\color{incolor}In [{\color{incolor}7}]:} \PY{k+kn}{from} \PY{n+nn}{utils} \PY{k}{import} \PY{n}{line\PYZus{}count}
\end{Verbatim}


    \begin{Verbatim}[commandchars=\\\{\}]
{\color{incolor}In [{\color{incolor}8}]:} \PY{n}{help}\PY{p}{(}\PY{n}{line\PYZus{}count}\PY{p}{)}
\end{Verbatim}


    \begin{Verbatim}[commandchars=\\\{\}]
Help on function line\_count in module utils:

line\_count(file)
    Computes the number of lines in a file.
    
    file: the file in which to count the lines.
    return: The number of lines in the file


    \end{Verbatim}

    \begin{Verbatim}[commandchars=\\\{\}]
{\color{incolor}In [{\color{incolor}9}]:} \PY{n}{line\PYZus{}count}\PY{p}{(}\PY{l+s+s1}{\PYZsq{}}\PY{l+s+s1}{data/Ecole Bilingue de Berkeley (37.854830799999995 \PYZhy{}122.28937169999999) Primary 08\PYZus{}05\PYZus{}2018 09\PYZus{}04\PYZus{}2018.csv}\PY{l+s+s1}{\PYZsq{}}\PY{p}{)}
\end{Verbatim}


\begin{Verbatim}[commandchars=\\\{\}]
{\color{outcolor}Out[{\color{outcolor}9}]:} 29894
\end{Verbatim}
            
    \begin{Verbatim}[commandchars=\\\{\}]
{\color{incolor}In [{\color{incolor}10}]:} \PY{k+kn}{from} \PY{n+nn}{utils} \PY{k}{import} \PY{n}{head}
\end{Verbatim}


    \begin{Verbatim}[commandchars=\\\{\}]
{\color{incolor}In [{\color{incolor}11}]:} \PY{n}{head}\PY{p}{(}\PY{l+s+s1}{\PYZsq{}}\PY{l+s+s1}{data/Ecole Bilingue de Berkeley (37.854830799999995 \PYZhy{}122.28937169999999) Primary 08\PYZus{}05\PYZus{}2018 09\PYZus{}04\PYZus{}2018.csv}\PY{l+s+s1}{\PYZsq{}}\PY{p}{)}
\end{Verbatim}


\begin{Verbatim}[commandchars=\\\{\}]
{\color{outcolor}Out[{\color{outcolor}11}]:} ['created\_at,entry\_id,PM1.0\_CF\_ATM\_ug/m3,PM2.5\_CF\_ATM\_ug/m3,PM10.0\_CF\_ATM\_ug/m3,UptimeMinutes,RSSI\_dbm,Temperature\_F,Humidity\_\%,PM2.5\_CF\_1\_ug/m3,\textbackslash{}n',
          '2018-08-05 00:00:31 UTC,111170,1.96,4.34,4.96,135.00,-67.00,84.00,33.00,4.34\textbackslash{}n',
          '2018-08-05 00:01:51 UTC,111171,2.13,3.89,6.83,136.00,-67.00,84.00,33.00,3.89\textbackslash{}n',
          '2018-08-05 00:03:11 UTC,111172,3.04,4.93,6.18,137.00,-68.00,84.00,34.00,4.93\textbackslash{}n',
          '2018-08-05 00:04:31 UTC,111173,2.17,4.26,6.83,139.00,-65.00,84.00,33.00,4.26\textbackslash{}n']
\end{Verbatim}
            
    This confirms the file type is .csv, so let's pull it in:

    \begin{Verbatim}[commandchars=\\\{\}]
{\color{incolor}In [{\color{incolor}12}]:} \PY{n}{EB\PYZus{}primary} \PY{o}{=} \PY{n}{pd}\PY{o}{.}\PY{n}{read\PYZus{}csv}\PY{p}{(}\PY{l+s+s1}{\PYZsq{}}\PY{l+s+s1}{data/Ecole Bilingue de Berkeley (37.854830799999995 \PYZhy{}122.28937169999999) Primary 08\PYZus{}05\PYZus{}2018 09\PYZus{}04\PYZus{}2018.csv}\PY{l+s+s1}{\PYZsq{}}\PY{p}{)}
         \PY{n}{EB\PYZus{}primary}\PY{o}{.}\PY{n}{head}\PY{p}{(}\PY{p}{)}
\end{Verbatim}


\begin{Verbatim}[commandchars=\\\{\}]
{\color{outcolor}Out[{\color{outcolor}12}]:}                 created\_at  entry\_id  PM1.0\_CF\_ATM\_ug/m3  PM2.5\_CF\_ATM\_ug/m3  \textbackslash{}
         0  2018-08-05 00:00:31 UTC    111170                1.96                4.34   
         1  2018-08-05 00:01:51 UTC    111171                2.13                3.89   
         2  2018-08-05 00:03:11 UTC    111172                3.04                4.93   
         3  2018-08-05 00:04:31 UTC    111173                2.17                4.26   
         4  2018-08-05 00:05:51 UTC    111174                2.06                4.06   
         
            PM10.0\_CF\_ATM\_ug/m3  UptimeMinutes  RSSI\_dbm  Temperature\_F  Humidity\_\%  \textbackslash{}
         0                 4.96          135.0     -67.0           84.0        33.0   
         1                 6.83          136.0     -67.0           84.0        33.0   
         2                 6.18          137.0     -68.0           84.0        34.0   
         3                 6.83          139.0     -65.0           84.0        33.0   
         4                 8.51          140.0     -67.0           84.0        33.0   
         
            PM2.5\_CF\_1\_ug/m3  Unnamed: 10  
         0              4.34          NaN  
         1              3.89          NaN  
         2              4.93          NaN  
         3              4.26          NaN  
         4              4.06          NaN  
\end{Verbatim}
            
    Several things to ask from this: 1. Dates are UTC. 2. Each entry has a
unique ID -\/- could be used to check for time stamp errors or gaps in
data 3. Headers have 'CF\_ATM' at the top -\/- what does that mean? 4.
There is one PM2.5 column without 'CF\_ATM', what is its significance?
1. From the PurpleAir documentation, in this directory, \emph{"ATM is
"atmospheric", meant to be used for outdoor applications. CF=1 is meant
to be used for indoor or controlled environment applications. However,
PurpleAir uses CF=1 values on the map. This value is lower than the ATM
value in higher measured concentrations."}\\
2. The explanation is a little vague and suggests further exploration
required. But it probably has something to do with how chaning
atmospheric pressure might change the measurements.\\
4. The columns "UptimeMinutes" and "RSSI\_dbm" are not immediately
obvious 1. again from documentation: "uptimeminutes" is time since last
restart, and "RSSI\_dbm" is wifi signal strength for the device.\\
5. The "unnamed: 10" column seems useless, why is it there? 1. Looking
at the data we see "\n" at the end of each line (newline character), it
appears this is generating the extra row.

    \subsection{Granularity}\label{granularity}

We'll talk a little more about Temporality in a moment, but time also
matters for thinking about granularity.

First we need to pay attention to the fact that this is UTC. L:et's put
it in datetime format to prevent mistakes.

    \begin{Verbatim}[commandchars=\\\{\}]
{\color{incolor}In [{\color{incolor}13}]:} \PY{n}{EB\PYZus{}time} \PY{o}{=} \PY{n}{pd}\PY{o}{.}\PY{n}{to\PYZus{}datetime}\PY{p}{(}\PY{n}{EB\PYZus{}primary}\PY{p}{[}\PY{l+s+s1}{\PYZsq{}}\PY{l+s+s1}{created\PYZus{}at}\PY{l+s+s1}{\PYZsq{}}\PY{p}{]}\PY{p}{,} \PY{n}{utc}\PY{o}{=}\PY{k+kc}{True}\PY{p}{)}
\end{Verbatim}


    \begin{Verbatim}[commandchars=\\\{\}]
{\color{incolor}In [{\color{incolor}14}]:} \PY{n}{EB\PYZus{}primary}\PY{p}{[}\PY{l+s+s1}{\PYZsq{}}\PY{l+s+s1}{created\PYZus{}at}\PY{l+s+s1}{\PYZsq{}}\PY{p}{]}\PY{o}{=}\PY{n}{EB\PYZus{}time}
\end{Verbatim}


    \begin{Verbatim}[commandchars=\\\{\}]
{\color{incolor}In [{\color{incolor}15}]:} \PY{n}{EB\PYZus{}primary}\PY{p}{[}\PY{l+s+s1}{\PYZsq{}}\PY{l+s+s1}{created\PYZus{}at}\PY{l+s+s1}{\PYZsq{}}\PY{p}{]}\PY{o}{.}\PY{n}{dtype}
\end{Verbatim}


\begin{Verbatim}[commandchars=\\\{\}]
{\color{outcolor}Out[{\color{outcolor}15}]:} datetime64[ns, UTC]
\end{Verbatim}
            
    \begin{Verbatim}[commandchars=\\\{\}]
{\color{incolor}In [{\color{incolor}16}]:} \PY{n}{EB\PYZus{}primary}\PY{o}{.}\PY{n}{head}\PY{p}{(}\PY{p}{)}
\end{Verbatim}


\begin{Verbatim}[commandchars=\\\{\}]
{\color{outcolor}Out[{\color{outcolor}16}]:}                  created\_at  entry\_id  PM1.0\_CF\_ATM\_ug/m3  PM2.5\_CF\_ATM\_ug/m3  \textbackslash{}
         0 2018-08-05 00:00:31+00:00    111170                1.96                4.34   
         1 2018-08-05 00:01:51+00:00    111171                2.13                3.89   
         2 2018-08-05 00:03:11+00:00    111172                3.04                4.93   
         3 2018-08-05 00:04:31+00:00    111173                2.17                4.26   
         4 2018-08-05 00:05:51+00:00    111174                2.06                4.06   
         
            PM10.0\_CF\_ATM\_ug/m3  UptimeMinutes  RSSI\_dbm  Temperature\_F  Humidity\_\%  \textbackslash{}
         0                 4.96          135.0     -67.0           84.0        33.0   
         1                 6.83          136.0     -67.0           84.0        33.0   
         2                 6.18          137.0     -68.0           84.0        34.0   
         3                 6.83          139.0     -65.0           84.0        33.0   
         4                 8.51          140.0     -67.0           84.0        33.0   
         
            PM2.5\_CF\_1\_ug/m3  Unnamed: 10  
         0              4.34          NaN  
         1              3.89          NaN  
         2              4.93          NaN  
         3              4.26          NaN  
         4              4.06          NaN  
\end{Verbatim}
            
    Nice thing about the datetime formate is that you can easily get time
information out of it:

    \begin{Verbatim}[commandchars=\\\{\}]
{\color{incolor}In [{\color{incolor}17}]:} \PY{n}{EB\PYZus{}primary}\PY{o}{.}\PY{n}{iloc}\PY{p}{[}\PY{l+m+mi}{1000}\PY{p}{,}\PY{l+m+mi}{0}\PY{p}{]}\PY{o}{.}\PY{n}{tzinfo}
\end{Verbatim}


\begin{Verbatim}[commandchars=\\\{\}]
{\color{outcolor}Out[{\color{outcolor}17}]:} <UTC>
\end{Verbatim}
            
    Note, we could rename the cols to make things easier if we wished. I'm
not going to because we're not going to be workign with this data set
for long, but in other cases you might decide to.

    Can we figure out how frequent measurements are?

Unfortunately I found it difficult to take differences with datetime
objects, so I had to write a for loop:

    \begin{Verbatim}[commandchars=\\\{\}]
{\color{incolor}In [{\color{incolor}18}]:} \PY{n}{diffs} \PY{o}{=} \PY{n}{np}\PY{o}{.}\PY{n}{zeros}\PY{p}{(}\PY{n+nb}{len}\PY{p}{(}\PY{n}{EB\PYZus{}primary}\PY{p}{[}\PY{l+s+s1}{\PYZsq{}}\PY{l+s+s1}{created\PYZus{}at}\PY{l+s+s1}{\PYZsq{}}\PY{p}{]}\PY{p}{)}\PY{p}{)}
         
         \PY{k}{for} \PY{n}{i} \PY{o+ow}{in} \PY{n+nb}{range}\PY{p}{(}\PY{l+m+mi}{0}\PY{p}{,} \PY{n+nb}{len}\PY{p}{(}\PY{n}{diffs}\PY{p}{)}\PY{o}{\PYZhy{}}\PY{l+m+mi}{1}\PY{p}{)}\PY{p}{:}
             \PY{n}{diffs}\PY{p}{[}\PY{n}{i}\PY{p}{]} \PY{o}{=} \PY{n+nb}{float}\PY{p}{(}\PY{p}{(}\PY{n}{EB\PYZus{}primary}\PY{p}{[}\PY{l+s+s1}{\PYZsq{}}\PY{l+s+s1}{created\PYZus{}at}\PY{l+s+s1}{\PYZsq{}}\PY{p}{]}\PY{p}{[}\PY{n}{i}\PY{o}{+}\PY{l+m+mi}{1}\PY{p}{]}\PY{o}{\PYZhy{}} \PY{n}{EB\PYZus{}primary}\PY{p}{[}\PY{l+s+s1}{\PYZsq{}}\PY{l+s+s1}{created\PYZus{}at}\PY{l+s+s1}{\PYZsq{}}\PY{p}{]}\PY{p}{[}\PY{n}{i}\PY{p}{]}\PY{p}{)}\PY{o}{.}\PY{n}{total\PYZus{}seconds}\PY{p}{(}\PY{p}{)}\PY{p}{)}
         
         \PY{n}{diffs} \PY{o}{=} \PY{n}{np}\PY{o}{.}\PY{n}{sort}\PY{p}{(}\PY{p}{(}\PY{n}{diffs}\PY{p}{)}\PY{p}{)}
         
         \PY{n+nb}{print}\PY{p}{(}\PY{l+s+s1}{\PYZsq{}}\PY{l+s+s1}{mins:}\PY{l+s+s1}{\PYZsq{}}\PY{p}{,} \PY{n}{diffs}\PY{p}{[}\PY{l+m+mi}{0}\PY{p}{:}\PY{l+m+mi}{30}\PY{p}{]}\PY{p}{)}
         \PY{n+nb}{print}\PY{p}{(}\PY{l+s+s1}{\PYZsq{}}\PY{l+s+s1}{maxes:}\PY{l+s+s1}{\PYZsq{}}\PY{p}{,} \PY{n}{diffs}\PY{p}{[}\PY{o}{\PYZhy{}}\PY{l+m+mi}{1}\PY{p}{:}\PY{o}{\PYZhy{}}\PY{l+m+mi}{30}\PY{p}{:}\PY{o}{\PYZhy{}}\PY{l+m+mi}{1}\PY{p}{]}\PY{p}{)}
         \PY{n+nb}{print}\PY{p}{(}\PY{l+s+s1}{\PYZsq{}}\PY{l+s+s1}{mean:}\PY{l+s+s1}{\PYZsq{}}\PY{p}{,} \PY{n}{np}\PY{o}{.}\PY{n}{mean}\PY{p}{(}\PY{n}{diffs}\PY{p}{)}\PY{p}{)}
\end{Verbatim}


    \begin{Verbatim}[commandchars=\\\{\}]
mins: [ 0. 14. 69. 69. 70. 70. 70. 70. 71. 71. 72. 72. 72. 72. 73. 73. 73. 73.
 73. 74. 74. 74. 74. 74. 74. 74. 75. 75. 75. 75.]
maxes: [134161.    931.    486.    481.    403.    384.    338.    334.    325.
    322.    320.    320.    320.    320.    317.    315.    306.    299.
    298.    256.    252.    245.    241.    241.    240.    240.    240.
    240.    240.]
mean: 86.70581741544844

    \end{Verbatim}

    Looks like for the most part we're sampling every 1.5 minutes or so,
with a few gaps in the data.

    \subsection{Scope}\label{scope}

This is data from one location -\/- French School in Berkeley.

From the file name it looks like the time is from early August to early
September, let's confirm:

    \begin{Verbatim}[commandchars=\\\{\}]
{\color{incolor}In [{\color{incolor}19}]:} \PY{n}{EB\PYZus{}primary}\PY{p}{[}\PY{l+s+s1}{\PYZsq{}}\PY{l+s+s1}{created\PYZus{}at}\PY{l+s+s1}{\PYZsq{}}\PY{p}{]}\PY{o}{.}\PY{n}{min}\PY{p}{(}\PY{p}{)}
\end{Verbatim}


\begin{Verbatim}[commandchars=\\\{\}]
{\color{outcolor}Out[{\color{outcolor}19}]:} Timestamp('2018-08-05 00:00:31+0000', tz='UTC')
\end{Verbatim}
            
    \begin{Verbatim}[commandchars=\\\{\}]
{\color{incolor}In [{\color{incolor}20}]:} \PY{n}{EB\PYZus{}primary}\PY{p}{[}\PY{l+s+s1}{\PYZsq{}}\PY{l+s+s1}{created\PYZus{}at}\PY{l+s+s1}{\PYZsq{}}\PY{p}{]}\PY{o}{.}\PY{n}{max}\PY{p}{(}\PY{p}{)}
\end{Verbatim}


\begin{Verbatim}[commandchars=\\\{\}]
{\color{outcolor}Out[{\color{outcolor}20}]:} Timestamp('2018-09-03 23:58:48+0000', tz='UTC')
\end{Verbatim}
            
    So it's about one month of data.

Does the data cover the topic of interest?

In this case, we need to answer the question: For the PurpleAir data,
what topic of interest might the data cover?

-\/-\textgreater{} class discussion on this.

    \subsection{Temporality}\label{temporality}

We've already figured out that we're working with UTC dates. UTC is
"universal time coordinated" and is essentially greenwich mean time, the
time on the prime meridian.

    \subsection{Faithfulness}\label{faithfulness}

This one's much harder to assess. Let's have a look at some basic things
we might care about

    \begin{Verbatim}[commandchars=\\\{\}]
{\color{incolor}In [{\color{incolor}21}]:} \PY{n+nb}{sum}\PY{p}{(}\PY{n}{EB\PYZus{}primary}\PY{p}{[}\PY{l+s+s1}{\PYZsq{}}\PY{l+s+s1}{PM2.5\PYZus{}CF\PYZus{}ATM\PYZus{}ug/m3}\PY{l+s+s1}{\PYZsq{}}\PY{p}{]}\PY{o}{.}\PY{n}{isna}\PY{p}{(}\PY{p}{)}\PY{p}{)}
\end{Verbatim}


\begin{Verbatim}[commandchars=\\\{\}]
{\color{outcolor}Out[{\color{outcolor}21}]:} 0
\end{Verbatim}
            
    That tells us there are no NaN values in the PM2.5 data. Impressive!

    \begin{Verbatim}[commandchars=\\\{\}]
{\color{incolor}In [{\color{incolor}22}]:} \PY{n}{EB\PYZus{}primary}\PY{o}{.}\PY{n}{describe}\PY{p}{(}\PY{p}{)}
\end{Verbatim}


\begin{Verbatim}[commandchars=\\\{\}]
{\color{outcolor}Out[{\color{outcolor}22}]:}             entry\_id  PM1.0\_CF\_ATM\_ug/m3  PM2.5\_CF\_ATM\_ug/m3  \textbackslash{}
         count   29893.000000        29893.000000        29893.000000   
         mean   126116.000000           15.656506           23.983548   
         std      8629.510135          120.382762          121.160645   
         min    111170.000000            0.330000            1.220000   
         25\%    118643.000000            5.040000            7.980000   
         50\%    126116.000000            9.510000           14.790000   
         75\%    133589.000000           17.280000           28.590000   
         max    141062.000000         5003.890000         5003.890000   
         
                PM10.0\_CF\_ATM\_ug/m3  UptimeMinutes      RSSI\_dbm  Temperature\_F  \textbackslash{}
         count         29893.000000   29893.000000  29893.000000   29893.000000   
         mean             28.240713     810.441508    -64.379654      71.434048   
         std             121.630947    1126.457388     10.010075       4.550603   
         min               1.310000       1.000000    -79.000000      63.000000   
         25\%              10.430000     183.000000    -67.000000      68.000000   
         50\%              18.540000     459.000000    -65.000000      70.000000   
         75\%              32.890000     960.000000    -63.000000      75.000000   
         max            5003.890000    6761.000000     31.000000      88.000000   
         
                  Humidity\_\%  PM2.5\_CF\_1\_ug/m3  Unnamed: 10  
         count  29893.000000      29893.000000          0.0  
         mean      49.988726         21.014233          NaN  
         std        6.149002         80.844029          NaN  
         min       27.000000          1.220000          NaN  
         25\%       45.000000          7.980000          NaN  
         50\%       52.000000         14.790000          NaN  
         75\%       55.000000         28.400000          NaN  
         max       61.000000       3335.440000          NaN  
\end{Verbatim}
            
    That's a pretty high PM2.5 average. And the max is very suspiciously
high. What's going on?

Options: 1. Wildfire smoke really pumped up the 2.5 values 2. We have a
lot of missing data and only values during the wild fires 3. There are
some erroneously high values.

    Let's start by looking at how many values are big.

    \begin{Verbatim}[commandchars=\\\{\}]
{\color{incolor}In [{\color{incolor}23}]:} \PY{n}{log\PYZus{}ind} \PY{o}{=} \PY{n}{EB\PYZus{}primary}\PY{o}{.}\PY{n}{loc}\PY{p}{[}\PY{p}{:}\PY{p}{,}\PY{l+s+s1}{\PYZsq{}}\PY{l+s+s1}{PM2.5\PYZus{}CF\PYZus{}ATM\PYZus{}ug/m3}\PY{l+s+s1}{\PYZsq{}}\PY{p}{]} \PY{o}{\PYZgt{}} \PY{l+m+mi}{150}
         \PY{n}{EB\PYZus{}primary}\PY{o}{.}\PY{n}{loc}\PY{p}{[}\PY{n}{log\PYZus{}ind}\PY{p}{,}\PY{l+s+s1}{\PYZsq{}}\PY{l+s+s1}{PM2.5\PYZus{}CF\PYZus{}ATM\PYZus{}ug/m3}\PY{l+s+s1}{\PYZsq{}}\PY{p}{]}
\end{Verbatim}


\begin{Verbatim}[commandchars=\\\{\}]
{\color{outcolor}Out[{\color{outcolor}23}]:} 2569     797.96
         2570     228.62
         2571     256.98
         9632    1497.30
         9633    5003.89
         9634    4999.74
         9635    4998.48
         9636    5000.70
         9637    5000.00
         9638    5000.00
         9639    5000.00
         9640    5000.00
         9641    5000.00
         9642    4998.30
         9643    4996.82
         9644    4998.23
         9645    4992.67
         9646    4998.00
         9647    5000.00
         9648    4996.89
         9649    4996.04
         9650    2187.27
         9692     201.70
         Name: PM2.5\_CF\_ATM\_ug/m3, dtype: float64
\end{Verbatim}
            
    Looks like there was a stretch of time with really high values, somewhat
suspciously clustered around 5000. If I were doing more work here I
would look into the sensor more carefully to see if there is any
significance to that number.

But for now -\/- let's just go ahead and drop them and see what happens:

    \begin{Verbatim}[commandchars=\\\{\}]
{\color{incolor}In [{\color{incolor}24}]:} \PY{n}{EB\PYZus{}primary}\PY{o}{.}\PY{n}{loc}\PY{p}{[}\PY{n}{log\PYZus{}ind}\PY{p}{,}\PY{l+s+s1}{\PYZsq{}}\PY{l+s+s1}{PM2.5\PYZus{}CF\PYZus{}ATM\PYZus{}ug/m3}\PY{l+s+s1}{\PYZsq{}}\PY{p}{]} \PY{o}{=} \PY{n}{np}\PY{o}{.}\PY{n}{nan}
         \PY{n}{EB\PYZus{}primary}\PY{o}{.}\PY{n}{describe}\PY{p}{(}\PY{p}{)}
\end{Verbatim}


\begin{Verbatim}[commandchars=\\\{\}]
{\color{outcolor}Out[{\color{outcolor}24}]:}             entry\_id  PM1.0\_CF\_ATM\_ug/m3  PM2.5\_CF\_ATM\_ug/m3  \textbackslash{}
         count   29893.000000        29893.000000        29870.000000   
         mean   126116.000000           15.656506           20.983950   
         std      8629.510135          120.382762           18.510573   
         min    111170.000000            0.330000            1.220000   
         25\%    118643.000000            5.040000            7.980000   
         50\%    126116.000000            9.510000           14.765000   
         75\%    133589.000000           17.280000           28.550000   
         max    141062.000000         5003.890000          115.950000   
         
                PM10.0\_CF\_ATM\_ug/m3  UptimeMinutes      RSSI\_dbm  Temperature\_F  \textbackslash{}
         count         29893.000000   29893.000000  29893.000000   29893.000000   
         mean             28.240713     810.441508    -64.379654      71.434048   
         std             121.630947    1126.457388     10.010075       4.550603   
         min               1.310000       1.000000    -79.000000      63.000000   
         25\%              10.430000     183.000000    -67.000000      68.000000   
         50\%              18.540000     459.000000    -65.000000      70.000000   
         75\%              32.890000     960.000000    -63.000000      75.000000   
         max            5003.890000    6761.000000     31.000000      88.000000   
         
                  Humidity\_\%  PM2.5\_CF\_1\_ug/m3  Unnamed: 10  
         count  29893.000000      29893.000000          0.0  
         mean      49.988726         21.014233          NaN  
         std        6.149002         80.844029          NaN  
         min       27.000000          1.220000          NaN  
         25\%       45.000000          7.980000          NaN  
         50\%       52.000000         14.790000          NaN  
         75\%       55.000000         28.400000          NaN  
         max       61.000000       3335.440000          NaN  
\end{Verbatim}
            

    % Add a bibliography block to the postdoc
    
    
    
    \end{document}
